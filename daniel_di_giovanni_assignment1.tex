\documentclass[12pt]{article}

% Math package
% leqno means numbered equations are displayed with the numbers to the left of
% the equations instead of to the right ("left equation numbers")
\usepackage[leqno]{amsmath}

% Geometry package for setting margins
\usepackage{geometry}
% Set margins
\geometry{margin=1in}

% Color package for highlighting text and array columns
\usepackage[table]{xcolor}

% Command \T now inserts a bold math t (for convenience)
\newcommand{\T}{\mathbf{t}}
% Command \F now inserts a bold math f (for convenience)
\newcommand{\F}{\mathbf{f}}

% Package for sans-serif font
\usepackage{helvet}
\renewcommand{\familydefault}{\sfdefault}

% Package for the Aboxed command
\usepackage{mathtools}

% Package for creating equations side-by-side
\usepackage{multicol}

% Package for adding space between paragraphs
\usepackage[skip=10pt plus1pt, indent=20pt]{parskip}

% Package for changing enumerate letters
\usepackage{enumitem}

% Adding "Page {number} of {total}" in footer
\usepackage{lastpage}
\usepackage{fancyhdr}
\fancyfoot[C]{
    Page
    \thepage\
    of
    {\hypersetup{linkcolor=black}\pageref{LastPage}}
}
\pagestyle{fancy}
% Remove horizontal line from header
\renewcommand{\headrulewidth}{0pt}

% I'm more used to typing \infin for infinity symbol :P
\newcommand*{\infin}{\infty}

% Colors
\usepackage{xcolor}
\definecolor{codegreen}{rgb}{0,0.6,0}
\definecolor{codegray}{rgb}{0.5,0.5,0.5}
\definecolor{codepurple}{rgb}{0.58,0,0.82}
\definecolor{backcolour}{rgb}{0.92,0.92,0.92}
\definecolor{rulecolour}{rgb}{0.5,0.5,0.5}

% For code blocks
\usepackage{listings}
\lstdefinestyle{mystyle}{
    backgroundcolor=\color{backcolour},
    commentstyle=\color{codegreen},
    keywordstyle=\color{magenta},
    numberstyle=\tiny\color{black},
    stringstyle=\color{codepurple},
    basicstyle=\ttfamily\footnotesize,
    rulecolor=\color{black},
    breakatwhitespace=false,
    breaklines=true,
    captionpos=b,
    keepspaces=true,
    numbers=left,
    numbersep=5pt,
    showspaces=false,
    showstringspaces=false,
    showtabs=false,
    tabsize=2
}
\lstset{style=mystyle}

% Package for diagonal fractions
\usepackage{xfrac}

% Floating images
\usepackage{float}

% Package for making the "Figure #:" bold in figure captions
\usepackage{caption}
\captionsetup{labelfont=bf}

% Hyperlinks
\usepackage{hyperref}
\hypersetup{
    colorlinks,
    citecolor=black,
    filecolor=black,
    linkcolor=black,
    urlcolor=black
}

% Package for table of contents
\usepackage{tocloft}
% Fill table of contents lines with dots
\renewcommand{\cftsecleader}{\cftdotfill{\cftdotsep}}
% --- Setting the spacing between lines in table of contents ---
% Spacing after section
\renewcommand\cftsecafterpnum{\vskip15pt}
% Spacing after subsection
\renewcommand\cftsubsecafterpnum{\vskip15pt}
% Spacing after subsubsection
\renewcommand\cftsubsubsecafterpnum{\vskip15pt}

% Appendix package
\usepackage[title, titletoc]{appendix}

% Shortcut command for making a link with the link as the text
\newcommand*{\urllink}[1]{\href{#1}{#1}}

% Shortcut command for "Figure #"
\newcommand*{\fig}[1]{\hyperref[fig:#1]{Figure~\ref{fig:#1}}}

\bibliographystyle{ieeetr}

\begin{document}

    % ----- Title page -----
    \newgeometry{left=1.3in, right=1.3in}
\newlength\myheight
\newlength\mydepth
\settototalheight\myheight{Xygp}
\settodepth\mydepth{Xygp}
\setlength\fboxsep{0pt}
\setlength{\fboxrule}{0pt}
\begin{titlepage}
    \Huge\noindent\textbf{EECS 3221 Report}

    \Large\noindent\textbf{A Comparison of Real Time Operating Systems and the Linux Operating System}

    \Large\noindent Daniel Di Giovanni --- 218204818

    \large\noindent\today

    \vfill

    \normalsize
    \begin{center}
        \rule{\textwidth}{1pt}\\
    \end{center}

    \noindent\textbf{My signature below attests that this submission is my original work:}
        % Make sure this line is justified
        \unskip\parfillskip 0pt \par

    \small\noindent
        Following professional engineering practice, I bear the burden of
            proof for original work. I have read the
            \href
                {https://www.yorku.ca/secretariat/policies/policies/academic-honesty-senate-policy-on/}
                {\color
                    {red}
                    {York University Senate Policy on Academic Integrity}
                }
            and the
            \href
                {http://www.cse.yorku.ca/admin/coscOnAcadHonesty.html}
                {\color
                    {red}
                    {EECS Academic Honesty Guidelines}
                }
            and confirm that this work is in accordance with the Policy.
    \vspace*{0.5cm}
    \noindent

    \begin{center}
        \begin{tabular}{ll}
            \raisebox{-30pt}{\fbox{\includegraphics[height=30pt]{images/singature.jpeg}}} & \raisebox{-30pt}{\fbox{\today}}\\
            \makebox[2.5in]{\hrulefill} & \makebox[2.5in]{\hrulefill}\\
            \textbf{Signature} & \textbf{Date}\\
        \end{tabular}
    \end{center}
\end{titlepage}
\restoregeometry


    % Numbering i, ii, iii, ...
    \pagenumbering{roman}

    % ----- Table of contents -----
    {\hypersetup{linkcolor=black}
    \tableofcontents
    \thispagestyle{plain}
}


    % Number 1, 2, 3, ...
    \pagenumbering{arabic}

    % Remove header text
    \markboth{}{}

    % ===== Sections =====

    % ----- Introduction -----
    \section*{Introduction and Background}
\addcontentsline{toc}{section}{Introduction and Background}
    An operating system is a computing layer that separates the hardware of the
        computer from the programs that run on it.
    It provides the \textit{environment} for other programs to do useful work
        \cite[p. 4]{textbook}.
    The fundamental tasks of an operating system include allocating resources
        (such as memory and CPU time), handling the control of input/output
        (I/O) devices, and ensuring proper usage of the computer and preventing
        errors \cite[pp. 3-5]{textbook}.

    The most important part of an operating system is the \textit{kernel}. It is
        the first program loaded into memory on startup and is the one program
        that is always running on the computer \cite[pp. 6-7, 22]{textbook}.
    Along with the kernel, operating systems also include
        \textit{middleware frameworks} that ease application development, and
        \textit{system programs} that help the system run but are not part of
        the ever-running kernel.
    All of this supports the execution of \textit{application programs}, which
        are the programs that provide functionality to the end user
        \cite[p. 4, 7]{textbook}.

    In industrial and commercial computing applications, the choice of an
        operating system is crucial.
    It affects the performance, security, and maintainability of the system.
    As an example, consider the secure boot of an embedded system.
    Secure boot, an important security technique to ensure that the kernel code
        has not been modified, is often neglected in embedded systems.
    The absence of secure boot allows the system to boot faster with less memory
        and energy consumption---at the cost of leaving the boot process and
        internal software vulnerable.
    However, it was discovered that the introduction of secure boot software
        caused boot-up time to increase by only 4\%, whereas a hardware
        implementation of secure boot caused a 36\% increase
        \cite[pp. 11-12]{ingelhag}.
    Clearly, the operating system has a significant impact on the overall
        quality of the system.

    Two important classes of operating systems/kernels will be discussed
        here: the Linux operating system and real-time operating systems (RTOSs).
    The Linux kernel is a free and open source implementation of an operating
        system kernel.
    It is used ubiquitously not only for desktop computers, but also for servers
        and embedded devices with a broad range of commercial and
        industrial applications \cite{whatislinux}.
    The Linux kernel is a tried-and-tested system with high flexibility and
        extendability.
    RTOSs are more vague, being defined not by a specific implementation, but by
        the ability to manage systems with complex time and resource
        constraints \cite{rtos-overview}.
    RTOSs need to be able to meet strict deadlines associated with external
        events using limited resources.
    In short, ``a real-time system is one whose correctness involves both the
        logical correctness of the outputs and their timeliness''
        \cite{laplante}.

    The objective of this report is to provide a thorough comparison of the
        Linux operating system/kernel with RTOS/real-time kernels to aid in the
        decision of which operating system to use.


    \section*{Overview of Linux}
\addcontentsline{toc}{section}{Overview of Linux}


    \section*{Overview of Real-Time Operating Systems}
\addcontentsline{toc}{section}{Overview of Real-Time Operating Systems}


    \section{\sloppy Comparing Linux-Based and Real-Time Operating Systems}
% \section*{\sloppy Comparing Linux-Based and Real-Time Operating Systems}
% \addcontentsline{toc}{section}{Comparing Linux-Based and Real-Time Operating Systems}
% Remove header text
\markboth{}{}
    RTOSs provide safety and predictability to the system they support.
    The ability to meet deadlines when responding to external events is
        necessary for hard real-time systems and crucial for soft ones.
    The price for this assurance is increased complexity, cost, and development
        time.
    Alternatively, Linux-based operating systems feature a suite of tools for
        handling soft real-time requirements.
    And, by virtue of the Linux kernel being open source, it can be extended to
        further meet real-time requirements by developers with expert Linux
        knowledge.
    Deciding whether to switch from a Linux distribution to a RTOS involves
        analyzing the benefits of a RTOS and balancing them with their
        associated costs.
    These benefits and tradeoffs between will be discussed here.
    But first, it must be acknowledged that there is a high degree of overlap
        between the two, especially regarding soft real-time systems.

        \subsection{The Overlap Between Linux-Based and Real-Time Operating Systems}
        % \subsection*{The Overlap Between Linux-Based and Real-Time Operating Systems}
        % \addcontentsline{toc}{subsection}{The Overlap Between Linux-Based and Real-Time Operating Systems}
        % Remove header text
        \markboth{}{}
            When utilizing Linux for real-time systems, there are generally two
                options.
            The first is to use the real-time capabilities built into the Linux
                kernel.
            One such capability is a patch to the kernel that makes it fully
                preemptible.
            The second option is to use a real-time framework for the Linux
                kernel.
            Three such frameworks are discussed, all of which function by adding
                a \textit{co-kernel} in a addition to the Linux kernel.

            \subsubsection{The Fully Preemptible Linux Kernel}
            % \subsubsection*{The Fully Preemptible Linux Kernel}
            % \addcontentsline{toc}{subsubsection}{The Fully Preemptible Linux Kernel}
                The real-time features of the Linux kernel revolve around
                    \textit{preemption}---pausing a running process to run a
                    higher-priority process instead.
                In user-mode, any process can be preempted, transferring control
                    to the kernel for scheduling \cite{ubuntu-real-time-part2}.
                The preemption strategy for kernel mode is where the potential
                    for real-time Linux lies.
                Mainline Linux (the vanilla version of the Linux kernel not
                    associated with any specific distribution) offers three
                    preemption strategies for the kernel:
                \begin{itemize}
                    \item
                        \textbf{PREEMPT\_NONE} forbids preemption when in
                            kernel mode; system call returns and interrupts are
                            the only preemption points
                            \cite{linux-preemption-models}.
                    \item
                        \textbf{PREEMPT\_VOLUNTARY} allows low-priority
                            processes to voluntarily preempt themselves when
                            executing a system call in kernel mode
                            \cite{ubuntu-real-time-part3}.
                    \item
                        \textbf{PREEMPT} makes all kernel code preemptible,
                        except for critical sections
                        \cite{linux-preemption-models}.
                \end{itemize}

                The fully preemptible kernel (the \textbf{PREEMPT\_RT} patch)
                    takes preemption further.
                This patch aimed to give Linux even more real-time capabilities.
                Although not part of mainline Linux, this patch is backed by the
                    Linux Foundation and is in use for real-time systems
                    \cite{rt-linux-riscv}.
                This preemption strategy further increases the number of
                    preemption points in kernel code and uses more real-time
                    data structures when handling interrupts and threads
                    \cite{linux-preemption-models}.

                The fully preemptible Linux kernel can be utilized in
                    high-performance computing and embedded industrial
                    environments.
                For complex real-time systems this kernel configuration provides
                    real-time capabilities within the familiarity of a
                    Linux-based operating system.
                The primary caveat is that there is no formal guarantee of worst
                    case execution times, and thus cannot be used for
                    safety-critical real-time systems \cite{preempt-rt-survey}.

            \subsubsection{Real-Time Linux Frameworks}
            % \subsubsection*{Real-Time Linux Frameworks}
            % \addcontentsline{toc}{subsubsection}{Real-Time Linux Frameworks}
            % Remove header text
            \markboth{}{}
                Other than the fully preemptible Linux kernel, the most common
                    approach to adapting the Linux kernel to real-time
                    environments involves using a real-time framework that adds
                    a co-kernel to the Linux kernel.
                The idea is to have the co-kernel working as a layer between the
                    hardware and the Linux kernel.
                The co-kernel is responsible for catching hardware interrupts,
                    scheduling them as either real-time tasks or Linux tasks,
                    and guaranteeing that real-time tasks meet their deadline.
                Leftover CPU time is given back to the Linux kernel
                    \cite{preempt-rt-survey}.

                The most common open-source implementations of a real-time
                    co-kernel Linux framework are RTLinux, Xenomai, and the Real
                    Time Application Interface for Linux (RTAI).

                \begin{itemize}
                    \item
                        \textbf{RTLinux} runs the Linux kernel as a
                            fully-preemptible process \cite{preempt-rt-survey}.
                        It then intercepts all hardware interrupts and schedules
                            tasks.
                        Shown in \fig{rtlinux_structure}, hardware interrupts
                            not related to real-time events are passed to the
                            Linux kernel as software interrupts, whereas
                            real-time event interrupts are handled by the
                            appropriate real-time interrupt service routines
                            \cite{rt-linux-getting-started}.
                        \begin{figure}[H]
                            \centering
                            \includegraphics
                                [width=0.7\textwidth]
                                {images/rtlinux_structure.png}
                            \caption
                                [Structure of RTLinux]
                                {Structure of RTLinux \cite{rt-linux-getting-started}.}
                            \label{fig:rtlinux_structure}
                        \end{figure}
                    \item
                        \textbf{Xenomai} works by supplementing the Linux kernel
                            with a real-time kernel running side-by-side with
                            it \cite{xenomai-overview}.
                        The real-time core deals with all time-critical
                            tasks and has a higher priority than the Linux
                            kernel.
                        The Xenomai kernel and the Linux kernel communicate via
                            the
                            \textit{Adaptive Domain Environment for Operating System (ADEOS)},
                            which separate domains for both kernels
                            \cite{preempt-rt-survey}.
                        This is shown in \fig{xenomai_structure}.
                        \begin{figure}[H]
                            \centering
                            \includegraphics
                                [width=0.5\textwidth]
                                {images/xenomai_structure.jpg}
                            \caption
                                [Structure of Xenomai]
                                {Structure of Xenomai \cite{preempt-rt-survey}.}
                            \label{fig:xenomai_structure}
                        \end{figure}
                    \item
                        \textbf{RTAI} uses a hardware abstraction layer (RTHAL)
                            to get information from Linux and dispatch
                            interrupts \cite{xenomai-overview}.
                        This is displayed in \fig{rtai_structure}.
                        It has few dependencies to Linux, making it easy to
                            switch between versions of the Linux kernel
                            \cite{xenomai-overview}.
                        \begin{figure}[H]
                            \centering
                            \includegraphics
                                [width=0.5\textwidth]
                                {images/rtai_structure.jpg}
                            \caption
                                [Structure of RTAI]
                                {Structure of RTAI \cite{preempt-rt-survey}.}
                            \label{fig:rtai_structure}
                        \end{figure}
                \end{itemize}

            \subsubsection{Choosing Between Preemptible Kernel and Real-Time Frameworks}
            % \subsubsection*{Choosing Between Preemptible Kernel and Real-Time Frameworks}
            % \addcontentsline{toc}{subsubsection}{Choosing Between Preemptible Kernel and Real-Time Frameworks}
            % Remove header text
            \markboth{}{}
                The fully preemptible Linux kernel and real-time Linux kernel
                    frameworks both add real-time capabilities to the Linux
                    kernel.
                The fully preemptible kernel achieves this by allowing the
                    kernel to be preempted in kernel mode and utilizing
                    real-time data structures \cite{linux-preemption-models}.
                However, these modifications are not enough to accommodate the
                    needs of a hard real-time system \cite{preempt-rt-survey}.
                At its core, the PREEMPT\_RT patch is still the Linux kernel,
                    offering no guarantees on worst case execution time
                    \cite{preempt-rt-survey}.
                This approach should only be taken for soft real-time systems.
                In this use case, the most important advantage of using the
                    fully preemptible Linux kernel is the cost and speed of
                    development.
                PREEMPT\_RT Linux does not differ much from mainline Linux.
                This allows the wealth of code for drivers and libraries in
                    Linux to be reused---although they may need to be adapted to
                    fit real-time needs.
                Further, many software engineers are competent with Linux and
                    can handle developing with the PREEMPT\_RT patch
                    \cite{preempt-rt-survey}.

                When tighter deadlines are required, a co-kernel Linux framework
                    can be used.
                These frameworks do have the ability to handle hard real-time
                    requirements \cite{preempt-rt-survey}.
                The cost of this functionality is development time and
                    complexity.
                Co-kernel approaches require modifications of the Linux kernel
                    code, none of which are fully backed by the
                    Linux community.
                This also introduces a dependency on the specific Linux kernel
                    version being modified, often being an older version
                    \cite{preempt-rt-survey}.
                The complex interaction between the two kernels introduces a
                    level of complexity that needs to be handled by skilled
                    real-time developers, increasing development efforts
                    \cite{preempt-rt-survey}.
                As a result, real-time Linux frameworks should be used only when
                    necessary, preferring the fully preemptible kernel when
                    timing requirements allow it.


    \section{Conclusion}
% Remove header text
\markboth{}{}
% \section*{Conclusion}
% \addcontentsline{toc}{section}{Conclusion}
    Choosing to migrate from a Linux distribution to a real-time operating
        system is a complex and technical decision.
    One option is to stay with Linux and utilize the preemption strategies
        provided by the kernel.
    The fully preemptible Linux kernel offers the closest capability to
        real-time, but still cannot support hard real-time systems.
    Still staying with Linux, the kernel can be modified to handle hard
        real-time deadlines.
    The most widely-used way of implementing this is with a co-kernel approach.

    Leaving the domain of Linux, there are both commercial and open source RTOSs
        that are not based on the Linux kernel.
    Choosing between these involves a careful processes of weighing the costs
        and efforts of development, maintenance, and system requirements.
    This is a very specific decision that should be guided by technical efforts
        to determine exactly what the real-time requirements are, whether or not
        they can be handled by a soft real-time system, and a balancing of the
        associated costs.


    \pagebreak

    % Remove header text
\markboth{}{}

\begin{thebibliography}{99}
\addcontentsline{toc}{section}{References}

    % Remove header text (I don't know why I have to do this twice)
    \markboth{}{}

    \bibitem{textbook}
        A. Silberschatz, P. B. Galvin, G. Gagne,
        \textit{Operating System Concepts}, 10th ed.,
        John Wiley and Sons, Inc., 2018.

    \bibitem{ingelhag}
        J. Ingelhag,
        ``How to choose an operating system for an embedded system'',
        \"Orebro Universitet, 2023.
        Accessed February 9, 2024.
        [Online].
        Available: \urllink{https://www.diva-portal.org/smash/get/diva2:1773441/FULLTEXT01.pdf}.

    \bibitem{whatislinux}
        The Linux Foundation,
        ``What is Linux?,''
        \textit{The Linux Foundation}.
        Accessed February 7, 2024.
        [Online].
        Available: \urllink{https://www.linux.com/what-is-linux/}.

    \bibitem{rtos-overview}
        W. Cede\~no, P.A. Laplante.
        ``An Overview of Real-Time Operating Systems,''
        JALA: Journal of the Association for Laboratory Automation, 2007, ch. 12, sec. 1, pp. 40-45.
        Accessed February 7, 2024.
        [Online].
        Available: \urllink{https://doi.org/10.1016/j.jala.2006.10.016}.

    \bibitem{laplante}
        P. A. Laplante,
        ``Real-Time Systems Design and Analysis,'' 3rd ed.,
        Hoboken, NJ, Wiley, 2004, p. 505.
        Accessed February 7, 2024.
        [Online].
        Available: \urllink{https://doi.org/10.1002/0471648299.fmatter}.

    \bibitem{stallman}
        R. Stallman,
        ``Linux and the GNU System,''
        \textit{Free Software Foundation}.
        Accessed February 9, 2024.
        [Online].
        Available: \urllink{https://www.gnu.org/gnu/linux-and-gnu.en.html}.

    \bibitem{comparative-os}
        A. Adekotujo, A. Odumabo, A. Adedokun, O. Aiyeniko,
        ``A Comparative Study of Operating Systems: Case of Windows, UNIX, Linux, Mac, Android and iOS,''
        International Journal of Computer Applications, 2020.
        Accessed February 10, 2024.
        [Online].
        Available: \href{https://www.researchgate.net/profile/Adedoyin-Odumabo/publication/343013056\_A\_Comparative\_Study\_of\_Operating\_Systems\_Case\_of\_Windows\_UNIX\_Linux\_Mac\_Android\_and\_iOS/links/61f2b50a9a753545e2fe8300/A-Comparative-Study-of-Operating-Systems-Case-of-Windows-UNIX-Linux-Mac-Android-and-iOS.pdf}{https://www.researchgate.net/profile/Adedoyin-Odumabo/publication/343013056\_A\_Comparative\_Study\_of\_Operating\_Systems\_Cas e\_of\_Windows\_UNIX\_Linux\_Mac\_Android\_and\_iOS/links/61f2b50a9a753545e2fe8300 /A-Comparative-Study-of-Operating-Systems-Case-of-Windows-UNIX-Linux-Mac-A ndroid-and-iOS.pdf}.

    \bibitem{grandview}
        Grand View Research,
        ``Server Operating System Market Size, Share \& Trends Analysis Report By Operating System (Windows, Linux), By Virtualization (Virtual Machine, Physical), By Deployment, By Region, And Segment Forecasts, 2022 - 2030,''
        \textit{Grand View Research}, 2020.
        Accessed February 10, 2024.
        [Online].
        Available: \urllink{https://www.grandviewresearch.com/industry-analysis/server-operating-system-market-report}.

    \bibitem{linux-public-cloud}
        The Linux Foundation,
        ``Linux Runs All of the World's Fastest Supercomputers,''
        \textit{The Linux Foundation}, November 20, 2017.
        Accessed February 10, 2024.
        [Online].
        Available: \urllink{https://www.linuxfoundation.org/blog/blog/linux-runs-all-of-the-worlds-fastest-supercomputers}.

    \bibitem{aws-linux}
        L. Clark,
        ``How Amazon Web Services Uses Linux and Open Source,''
        \textit{The Linux Foundation}, September 8, 2014.
        Accessed February 10, 2024.
        [Online].
        Available: \urllink{https://www.linux.com/news/how-amazon-web-services-uses-linux-and-open-source/}.

    \bibitem{bare-metal}
        M. Salehi, D. Hughes, B. Crispo,
        ``MicroGuard: Securing Bare-Metal Microcontrollers against Code-Reuse Attacks,''
        \textit{2019 IEEE Conference on Dependable and Secure Computing (DSC)},
        Hangzhou, China, IEEE, 2019, pp. 1-8, doi: 10.1109/DSC47296.2019.8937667.
        Accessed February 10, 2024.
        [Online].
        Available: \urllink{https://doi.org/10.1109/DSC47296.2019.8937667}.

    \bibitem{embedded-textbook}
        P. Raghavan, A. Lad, S. Neelakandan,
        \textit{Embedded Linux System Design and Development},
        Boca Raton, FL, Taylor and Francis Group, LLC, 2006.

    \bibitem{rtos-definition}
        A. S. Gillis,
        ``DEFINITION real-time operating system (RTOS),''
        TechTarget
        Accessed February 11, 2024.
        [Online].
        Available: \urllink{https://www.techtarget.com/searchdatacenter/definition/real-time-operating-system}

    \bibitem{intel-hard-soft-real-time}
        Intel,
        ``Real-Time Systems Overview,''
        Intel.
        Accessed February 11, 2024.
        [Online].
        Available: \urllink{https://www.intel.com/content/www/us/en/robotics/real-time-systems.html}

    \bibitem{hard-soft-real-time}
        G. Lipari, L. Palopoli,
        ``Real-Time scheduling: from hard to soft real-time systems,''
        arXiv, 2015.
        Accessed February 11, 2024.
        [Online].
        Available: \urllink{https://doi.org/10.48550/arXiv.1512.01978}

\end{thebibliography}


\end{document}
