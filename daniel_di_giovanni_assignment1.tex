\documentclass[12pt]{article}

% Math package
% leqno means numbered equations are displayed with the numbers to the left of
% the equations instead of to the right ("left equation numbers")
\usepackage[leqno]{amsmath}

% Geometry package for setting margins
\usepackage{geometry}
% Set margins
\geometry{margin=1in}

% Color package for highlighting text and array columns
\usepackage[table]{xcolor}

% Command \T now inserts a bold math t (for convenience)
\newcommand{\T}{\mathbf{t}}
% Command \F now inserts a bold math f (for convenience)
\newcommand{\F}{\mathbf{f}}

% Package for sans-serif font
\usepackage{helvet}
\renewcommand{\familydefault}{\sfdefault}

% Package for the Aboxed command
\usepackage{mathtools}

% Package for creating equations side-by-side
\usepackage{multicol}

% Package for adding space between paragraphs
\usepackage[skip=10pt plus1pt, indent=20pt]{parskip}

% Package for changing enumerate letters
\usepackage{enumitem}

% Adding "Page {number} of {total}" in footer
\usepackage{lastpage}
\usepackage{fancyhdr}
\fancyfoot[C]{
    Page
    \thepage\
    of
    {\hypersetup{linkcolor=black}\pageref{LastPage}}
}
\pagestyle{fancy}
% Remove horizontal line from header
\renewcommand{\headrulewidth}{0pt}

% I'm more used to typing \infin for infinity symbol :P
\newcommand*{\infin}{\infty}

% Colors
\usepackage{xcolor}
\definecolor{codegreen}{rgb}{0,0.6,0}
\definecolor{codegray}{rgb}{0.5,0.5,0.5}
\definecolor{codepurple}{rgb}{0.58,0,0.82}
\definecolor{backcolour}{rgb}{0.92,0.92,0.92}
\definecolor{rulecolour}{rgb}{0.5,0.5,0.5}

% For code blocks
\usepackage{listings}
\lstdefinestyle{mystyle}{
    backgroundcolor=\color{backcolour},
    commentstyle=\color{codegreen},
    keywordstyle=\color{magenta},
    numberstyle=\tiny\color{black},
    stringstyle=\color{codepurple},
    basicstyle=\ttfamily\footnotesize,
    rulecolor=\color{black},
    breakatwhitespace=false,
    breaklines=true,
    captionpos=b,
    keepspaces=true,
    numbers=left,
    numbersep=5pt,
    showspaces=false,
    showstringspaces=false,
    showtabs=false,
    tabsize=2
}
\lstset{style=mystyle}

% Package for diagonal fractions
\usepackage{xfrac}

% Floating images
\usepackage{float}

% Hyperlinks
\usepackage{hyperref}
\hypersetup{
    colorlinks,
    citecolor=black,
    filecolor=black,
    linkcolor=black,
    urlcolor=black
}

% Fill table of contents lines with dots
\usepackage{tocloft}
\renewcommand{\cftsecleader}{\cftdotfill{\cftdotsep}}

% Appendix package
\usepackage[title, titletoc]{appendix}

\newcommand*{\urllink}[1]{\href{#1}{#1}}

\begin{document}

    % ----- Title page -----
    \newgeometry{left=1.3in, right=1.3in}
\newlength\myheight
\newlength\mydepth
\settototalheight\myheight{Xygp}
\settodepth\mydepth{Xygp}
\setlength\fboxsep{0pt}
\setlength{\fboxrule}{0pt}
\begin{titlepage}
    \Huge\noindent\textbf{EECS 3221 Report}

    \Large\noindent\textbf{A Comparison of Real Time Operating Systems and the Linux Operating System}

    \Large\noindent Daniel Di Giovanni --- 218204818

    \large\noindent\today

    \vfill

    \normalsize
    \begin{center}
        \rule{\textwidth}{1pt}\\
    \end{center}

    \noindent\textbf{My signature below attests that this submission is my original work:}
        % Make sure this line is justified
        \unskip\parfillskip 0pt \par

    \small\noindent
        Following professional engineering practice, I bear the burden of
            proof for original work. I have read the
            \href
                {https://www.yorku.ca/secretariat/policies/policies/academic-honesty-senate-policy-on/}
                {\color
                    {red}
                    {York University Senate Policy on Academic Integrity}
                }
            and the
            \href
                {http://www.cse.yorku.ca/admin/coscOnAcadHonesty.html}
                {\color
                    {red}
                    {EECS Academic Honesty Guidelines}
                }
            and confirm that this work is in accordance with the Policy.
    \vspace*{0.5cm}
    \noindent

    \begin{center}
        \begin{tabular}{ll}
            \raisebox{-30pt}{\fbox{\includegraphics[height=30pt]{images/singature.jpeg}}} & \raisebox{-30pt}{\fbox{\today}}\\
            \makebox[2.5in]{\hrulefill} & \makebox[2.5in]{\hrulefill}\\
            \textbf{Signature} & \textbf{Date}\\
        \end{tabular}
    \end{center}
\end{titlepage}
\restoregeometry


    % Numbering i, ii, iii, ...
    \pagenumbering{roman}

    % ----- Table of contents -----
    {\hypersetup{linkcolor=black}
    \tableofcontents
    \thispagestyle{plain}
}


    % Number 1, 2, 3, ...
    \pagenumbering{arabic}

    % Remove header text
    \markboth{}{}

    % ===== Sections =====

    % ----- Introduction -----
    \section*{Introduction and Background}
\addcontentsline{toc}{section}{Introduction and Background}
    An operating system is a computing layer that separates the hardware of the
        computer from the programs that run on it.
    It provides the \textit{environment} for other programs to do useful work
        \cite[p. 4]{textbook}.
    The fundamental tasks of an operating system include allocating resources
        (such as memory and CPU time), handling the control of input/output
        (I/O) devices, and ensuring proper usage of the computer and preventing
        errors \cite[pp. 3-5]{textbook}.

    The most important part of an operating system is the \textit{kernel}. It is
        the first program loaded into memory on startup and is the one program
        that is always running on the computer \cite[pp. 6-7, 22]{textbook}.
    Along with the kernel, operating systems also include
        \textit{middleware frameworks} that ease application development, and
        \textit{system programs} that help the system run but are not part of
        the ever-running kernel.
    All of this supports the execution of \textit{application programs}, which
        are the programs that provide functionality to the end user
        \cite[p. 4, 7]{textbook}.

    In industrial and commercial computing applications, the choice of an
        operating system is crucial.
    It affects the performance, security, and maintainability of the system.
    As an example, consider the secure boot of an embedded system.
    Secure boot, an important security technique to ensure that the kernel code
        has not been modified, is often neglected in embedded systems.
    The absence of secure boot allows the system to boot faster with less memory
        and energy consumption---at the cost of leaving the boot process and
        internal software vulnerable.
    However, it was discovered that the introduction of secure boot software
        caused boot-up time to increase by only 4\%, whereas a hardware
        implementation of secure boot caused a 36\% increase
        \cite[pp. 11-12]{ingelhag}.
    Clearly, the operating system has a significant impact on the overall
        quality of the system.

    Two important classes of operating systems/kernels will be discussed
        here: the Linux operating system and real-time operating systems (RTOSs).
    The Linux kernel is a free and open source implementation of an operating
        system kernel.
    It is used ubiquitously not only for desktop computers, but also for servers
        and embedded devices with a broad range of commercial and
        industrial applications \cite{whatislinux}.
    The Linux kernel is a tried-and-tested system with high flexibility and
        extendability.
    RTOSs are more vague, being defined not by a specific implementation, but by
        the ability to manage systems with complex time and resource
        constraints \cite{rtos-overview}.
    RTOSs need to be able to meet strict deadlines associated with external
        events using limited resources.
    In short, ``a real-time system is one whose correctness involves both the
        logical correctness of the outputs and their timeliness''
        \cite{laplante}.

    The objective of this report is to provide a thorough comparison of the
        Linux operating system/kernel with RTOS/real-time kernels to aid in the
        decision of which operating system to use.


    \section*{Overview of Linux}
\addcontentsline{toc}{section}{Overview of Linux}


    \section*{Overview of Real-Time Operating Systems}
\addcontentsline{toc}{section}{Overview of Real-Time Operating Systems}


    \section*{Conclusion}
    \addcontentsline{toc}{section}{Conclusion}

    \pagebreak

    % Remove header text
\markboth{}{}

\begin{thebibliography}{99}
\addcontentsline{toc}{section}{References}

    % Remove header text (I don't know why I have to do this twice)
    \markboth{}{}

    \bibitem{textbook}
        A. Silberschatz, P. B. Galvin, G. Gagne,
        \textit{Operating System Concepts}, 10th ed.,
        John Wiley and Sons, Inc., 2018.

    \bibitem{ingelhag}
        J. Ingelhag,
        ``How to choose an operating system for an embedded system'',
        \"Orebro Universitet, 2023.
        Accessed February 9, 2024.
        [Online].
        Available: \urllink{https://www.diva-portal.org/smash/get/diva2:1773441/FULLTEXT01.pdf}.

    \bibitem{whatislinux}
        The Linux Foundation,
        ``What is Linux?,''
        \textit{The Linux Foundation}.
        Accessed February 7, 2024.
        [Online].
        Available: \urllink{https://www.linux.com/what-is-linux/}.

    \bibitem{rtos-overview}
        W. Cede\~no, P.A. Laplante.
        ``An Overview of Real-Time Operating Systems,''
        JALA: Journal of the Association for Laboratory Automation, 2007, ch. 12, sec. 1, pp. 40-45.
        Accessed February 7, 2024.
        [Online].
        Available: \urllink{https://doi.org/10.1016/j.jala.2006.10.016}.

    \bibitem{laplante}
        P. A. Laplante,
        ``Real-Time Systems Design and Analysis,'' 3rd ed.,
        Hoboken, NJ, Wiley, 2004, p. 505.
        Accessed February 7, 2024.
        [Online].
        Available: \urllink{https://doi.org/10.1002/0471648299.fmatter}.

    \bibitem{stallman}
        R. Stallman,
        ``Linux and the GNU System,''
        \textit{Free Software Foundation}.
        Accessed February 9, 2024.
        [Online].
        Available: \urllink{https://www.gnu.org/gnu/linux-and-gnu.en.html}.

    \bibitem{comparative-os}
        A. Adekotujo, A. Odumabo, A. Adedokun, O. Aiyeniko,
        ``A Comparative Study of Operating Systems: Case of Windows, UNIX, Linux, Mac, Android and iOS,''
        International Journal of Computer Applications, 2020.
        Accessed February 10, 2024.
        [Online].
        Available: \href{https://www.researchgate.net/profile/Adedoyin-Odumabo/publication/343013056\_A\_Comparative\_Study\_of\_Operating\_Systems\_Case\_of\_Windows\_UNIX\_Linux\_Mac\_Android\_and\_iOS/links/61f2b50a9a753545e2fe8300/A-Comparative-Study-of-Operating-Systems-Case-of-Windows-UNIX-Linux-Mac-Android-and-iOS.pdf}{https://www.researchgate.net/profile/Adedoyin-Odumabo/publication/343013056\_A\_Comparative\_Study\_of\_Operating\_Systems\_Cas e\_of\_Windows\_UNIX\_Linux\_Mac\_Android\_and\_iOS/links/61f2b50a9a753545e2fe8300 /A-Comparative-Study-of-Operating-Systems-Case-of-Windows-UNIX-Linux-Mac-A ndroid-and-iOS.pdf}.

    \bibitem{grandview}
        Grand View Research,
        ``Server Operating System Market Size, Share \& Trends Analysis Report By Operating System (Windows, Linux), By Virtualization (Virtual Machine, Physical), By Deployment, By Region, And Segment Forecasts, 2022 - 2030,''
        \textit{Grand View Research}, 2020.
        Accessed February 10, 2024.
        [Online].
        Available: \urllink{https://www.grandviewresearch.com/industry-analysis/server-operating-system-market-report}.

    \bibitem{linux-public-cloud}
        The Linux Foundation,
        ``Linux Runs All of the World's Fastest Supercomputers,''
        \textit{The Linux Foundation}, November 20, 2017.
        Accessed February 10, 2024.
        [Online].
        Available: \urllink{https://www.linuxfoundation.org/blog/blog/linux-runs-all-of-the-worlds-fastest-supercomputers}.

    \bibitem{aws-linux}
        L. Clark,
        ``How Amazon Web Services Uses Linux and Open Source,''
        \textit{The Linux Foundation}, September 8, 2014.
        Accessed February 10, 2024.
        [Online].
        Available: \urllink{https://www.linux.com/news/how-amazon-web-services-uses-linux-and-open-source/}.

    \bibitem{bare-metal}
        M. Salehi, D. Hughes, B. Crispo,
        ``MicroGuard: Securing Bare-Metal Microcontrollers against Code-Reuse Attacks,''
        \textit{2019 IEEE Conference on Dependable and Secure Computing (DSC)},
        Hangzhou, China, IEEE, 2019, pp. 1-8, doi: 10.1109/DSC47296.2019.8937667.
        Accessed February 10, 2024.
        [Online].
        Available: \urllink{https://doi.org/10.1109/DSC47296.2019.8937667}.

    \bibitem{embedded-textbook}
        P. Raghavan, A. Lad, S. Neelakandan,
        \textit{Embedded Linux System Design and Development},
        Boca Raton, FL, Taylor and Francis Group, LLC, 2006.

    \bibitem{rtos-definition}
        A. S. Gillis,
        ``DEFINITION real-time operating system (RTOS),''
        TechTarget
        Accessed February 11, 2024.
        [Online].
        Available: \urllink{https://www.techtarget.com/searchdatacenter/definition/real-time-operating-system}

    \bibitem{intel-hard-soft-real-time}
        Intel,
        ``Real-Time Systems Overview,''
        Intel.
        Accessed February 11, 2024.
        [Online].
        Available: \urllink{https://www.intel.com/content/www/us/en/robotics/real-time-systems.html}

    \bibitem{hard-soft-real-time}
        G. Lipari, L. Palopoli,
        ``Real-Time scheduling: from hard to soft real-time systems,''
        arXiv, 2015.
        Accessed February 11, 2024.
        [Online].
        Available: \urllink{https://doi.org/10.48550/arXiv.1512.01978}

\end{thebibliography}


    \begin{appendices}
        \section{Code}
            \subsection{Q1.m}
            \label{sec:q1}
    \end{appendices}

\end{document}
