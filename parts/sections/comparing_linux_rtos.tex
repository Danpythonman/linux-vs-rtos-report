\section{\sloppy Comparing Linux-Based and Real-Time Operating Systems}
% \section*{\sloppy Comparing Linux-Based and Real-Time Operating Systems}
% \addcontentsline{toc}{section}{Comparing Linux-Based and Real-Time Operating Systems}
% Remove header text
\markboth{}{}
    RTOSs provide safety and predictability to the system they support.
    The ability to meet deadlines when responding to external events is
        necessary for hard real-time systems and crucial for soft ones.
    The price for this assurance is increased complexity, cost, and development
        time.
    Alternatively, Linux-based operating systems feature a suite of tools for
        handling soft real-time requirements.
    And, by virtue of the Linux kernel being open source, it can be extended to
        further meet real-time requirements by developers with expert Linux
        knowledge.
    Deciding whether to switch from a Linux distribution to a RTOS involves
        analyzing the benefits of a RTOS and balancing them with their
        associated costs.
    These benefits and tradeoffs between will be discussed here.
    But first, it must be acknowledged that there is a high degree of overlap
        between the two, especially regarding soft real-time systems.

        \subsection{The Overlap Between Linux-Based and Real-Time Operating Systems}
        % \subsection*{The Overlap Between Linux-Based and Real-Time Operating Systems}
        % \addcontentsline{toc}{subsection}{The Overlap Between Linux-Based and Real-Time Operating Systems}
        % Remove header text
        \markboth{}{}
            When utilizing Linux for real-time systems, there are generally two
                options.
            The first is to use the real-time capabilities built into the Linux
                kernel.
            One such capability is a patch to the kernel that makes it fully
                preemptible.
            The second option is to use a real-time framework for the Linux
                kernel.
            Three such frameworks are discussed, all of which function by adding
                a \textit{co-kernel} in a addition to the Linux kernel.

            \subsubsection{The Fully Preemptible Linux Kernel}
            % \subsubsection*{The Fully Preemptible Linux Kernel}
            % \addcontentsline{toc}{subsubsection}{The Fully Preemptible Linux Kernel}
                The real-time features of the Linux kernel revolve around
                    \textit{preemption}---pausing a running process to run a
                    higher-priority process instead.
                In user-mode, any process can be preempted, transferring control
                    to the kernel for scheduling \cite{ubuntu-real-time-part2}.
                The preemption strategy for kernel mode is where the potential
                    for real-time Linux lies.
                Mainline Linux (the vanilla version of the Linux kernel not
                    associated with any specific distribution) offers three
                    preemption strategies for the kernel:
                \begin{itemize}
                    \item
                        \textbf{PREEMPT\_NONE} forbids preemption when in
                            kernel mode; system call returns and interrupts are
                            the only preemption points
                            \cite{linux-preemption-models}.
                    \item
                        \textbf{PREEMPT\_VOLUNTARY} allows low-priority
                            processes to voluntarily preempt themselves when
                            executing a system call in kernel mode
                            \cite{ubuntu-real-time-part3}.
                    \item
                        \textbf{PREEMPT} makes all kernel code preemptible,
                        except for critical sections
                        \cite{linux-preemption-models}.
                \end{itemize}

                The fully preemptible kernel (the \textbf{PREEMPT\_RT} patch)
                    takes preemption further.
                This patch aimed to give Linux even more real-time capabilities.
                Although not part of mainline Linux, this patch is backed by the
                    Linux Foundation and is in use for real-time systems
                    \cite{rt-linux-riscv}.
                This preemption strategy further increases the number of
                    preemption points in kernel code and uses more real-time
                    data structures when handling interrupts and threads
                    \cite{linux-preemption-models}.

                The fully preemptible Linux kernel can be utilized in
                    high-performance computing and embedded industrial
                    environments.
                For complex real-time systems this kernel configuration provides
                    real-time capabilities within the familiarity of a
                    Linux-based operating system.
                The primary caveat is that there is no formal guarantee of worst
                    case execution times, and thus cannot be used for
                    safety-critical real-time systems \cite{preempt-rt-survey}.

            \subsubsection{Real-Time Linux Frameworks}
            % \subsubsection*{Real-Time Linux Frameworks}
            % \addcontentsline{toc}{subsubsection}{Real-Time Linux Frameworks}
            % Remove header text
            \markboth{}{}
                Other than the fully preemptible Linux kernel, the most common
                    approach to adapting the Linux kernel to real-time
                    environments involves using a real-time framework that adds
                    a co-kernel to the Linux kernel.
                The idea is to have the co-kernel working as a layer between the
                    hardware and the Linux kernel.
                The co-kernel is responsible for catching hardware interrupts,
                    scheduling them as either real-time tasks or Linux tasks,
                    and guaranteeing that real-time tasks meet their deadline.
                Leftover CPU time is given back to the Linux kernel
                    \cite{preempt-rt-survey}.

                The most common open-source implementations of a real-time
                    co-kernel Linux framework are RTLinux, Xenomai, and the Real
                    Time Application Interface for Linux (RTAI).

                \begin{itemize}
                    \item
                        \textbf{RTLinux} runs the Linux kernel as a
                            fully-preemptible process \cite{preempt-rt-survey}.
                        It then intercepts all hardware interrupts and schedules
                            tasks.
                        Shown in \fig{rtlinux_structure}, hardware interrupts
                            not related to real-time events are passed to the
                            Linux kernel as software interrupts, whereas
                            real-time event interrupts are handled by the
                            appropriate real-time interrupt service routines
                            \cite{rt-linux-getting-started}.
                        \begin{figure}[H]
                            \centering
                            \includegraphics
                                [width=0.7\textwidth]
                                {images/rtlinux_structure.png}
                            \caption
                                [Structure of RTLinux]
                                {Structure of RTLinux \cite{rt-linux-getting-started}.}
                            \label{fig:rtlinux_structure}
                        \end{figure}
                    \item
                        \textbf{Xenomai} works by supplementing the Linux kernel
                            with a real-time kernel running side-by-side with
                            it \cite{xenomai-overview}.
                        The real-time core deals with all time-critical
                            tasks and has a higher priority than the Linux
                            kernel.
                        The Xenomai kernel and the Linux kernel communicate via
                            the
                            \textit{Adaptive Domain Environment for Operating System (ADEOS)},
                            which separate domains for both kernels
                            \cite{preempt-rt-survey}.
                        This is shown in \fig{xenomai_structure}.
                        \begin{figure}[H]
                            \centering
                            \includegraphics
                                [width=0.5\textwidth]
                                {images/xenomai_structure.jpg}
                            \caption
                                [Structure of Xenomai]
                                {Structure of Xenomai \cite{preempt-rt-survey}.}
                            \label{fig:xenomai_structure}
                        \end{figure}
                    \item
                        \textbf{RTAI} uses a hardware abstraction layer (RTHAL)
                            to get information from Linux and dispatch
                            interrupts \cite{xenomai-overview}.
                        This is displayed in \fig{rtai_structure}.
                        It has few dependencies to Linux, making it easy to
                            switch between versions of the Linux kernel
                            \cite{xenomai-overview}.
                        \begin{figure}[H]
                            \centering
                            \includegraphics
                                [width=0.5\textwidth]
                                {images/rtai_structure.jpg}
                            \caption
                                [Structure of RTAI]
                                {Structure of RTAI \cite{preempt-rt-survey}.}
                            \label{fig:rtai_structure}
                        \end{figure}
                \end{itemize}

            \subsubsection{Choosing Between Preemptible Kernel and Real-Time Frameworks}
            % \subsubsection*{Choosing Between Preemptible Kernel and Real-Time Frameworks}
            % \addcontentsline{toc}{subsubsection}{Choosing Between Preemptible Kernel and Real-Time Frameworks}
            % Remove header text
            \markboth{}{}
                The fully preemptible Linux kernel and real-time Linux kernel
                    frameworks both add real-time capabilities to the Linux
                    kernel.
                The fully preemptible kernel achieves this by allowing the
                    kernel to be preempted in kernel mode and utilizing
                    real-time data structures \cite{linux-preemption-models}.
                However, these modifications are not enough to accommodate the
                    needs of a hard real-time system \cite{preempt-rt-survey}.
                At its core, the PREEMPT\_RT patch is still the Linux kernel,
                    offering no guarantees on worst case execution time
                    \cite{preempt-rt-survey}.
                This approach should only be taken for soft real-time systems.
                In this use case, the most important advantage of using the
                    fully preemptible Linux kernel is the cost and speed of
                    development.
                PREEMPT\_RT Linux does not differ much from mainline Linux.
                This allows the wealth of code for drivers and libraries in
                    Linux to be reused---although they may need to be adapted to
                    fit real-time needs.
                Further, many software engineers are competent with Linux and
                    can handle developing with the PREEMPT\_RT patch
                    \cite{preempt-rt-survey}.

                When tighter deadlines are required, a co-kernel Linux framework
                    can be used.
                These frameworks do have the ability to handle hard real-time
                    requirements \cite{preempt-rt-survey}.
                The cost of this functionality is development time and
                    complexity.
                Co-kernel approaches require modifications of the Linux kernel
                    code, none of which are fully backed by the
                    Linux community.
                This also introduces a dependency on the specific Linux kernel
                    version being modified, often being an older version
                    \cite{preempt-rt-survey}.
                The complex interaction between the two kernels introduces a
                    level of complexity that needs to be handled by skilled
                    real-time developers, increasing development efforts
                    \cite{preempt-rt-survey}.
                As a result, real-time Linux frameworks should be used only when
                    necessary, preferring the fully preemptible kernel when
                    timing requirements allow it.
