\section{Conclusion}
% Remove header text
\markboth{}{}
% \section*{Conclusion}
% \addcontentsline{toc}{section}{Conclusion}
    Choosing to migrate from a Linux distribution to a real-time operating
        system is a complex and technical decision.
    One option is to stay with Linux and utilize the preemption strategies
        provided by the kernel.
    The fully preemptible Linux kernel offers the closest capability to
        real-time, but still cannot support hard real-time systems.
    Still staying with Linux, the kernel can be modified to handle hard
        real-time deadlines.
    The most widely-used way of implementing this is with a co-kernel approach.

    Leaving the domain of Linux, there are both commercial and open source RTOSs
        that are not based on the Linux kernel.
    Choosing between these involves a careful processes of weighing the costs
        and efforts of development, maintenance, and system requirements.
    This is a very specific decision that should be guided by technical efforts
        to determine exactly what the real-time requirements are, whether or not
        they can be handled by a soft real-time system, and a balancing of the
        associated costs.
