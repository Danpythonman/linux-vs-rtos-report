\section*{Overview of Linux}
\addcontentsline{toc}{section}{Overview of Linux}
    When ``Linux'' is referred to, an entire operating system is often being
        referenced.
    However, ``Linux'' is just the kernel.
    The Linux kernel is used in combination with other software to make a
    complete operating system.
    The entire operating system (with the Linux kernel inside) is called a
        ``Linux distribution'' (for example, Ubuntu and Debian for PCs)
        \cite{stallman}.
    The ability to extend and modify a Linux operating system is where its
        flexibility originates.

    Since Linux is an open source kernel, anyone can read, use, and modify the
        code.
    And since it is just a kernel, it \textit{requires} additional software to
        be useful.
    This leads to a wide variety of adaptations of the Linux-based operating
        systems to fit many different needs \cite{comparative-os}.

    Many Linux distributions are desktop-focused, creating an easy
        user-interface, similar to Microsoft's Windows and Apple's MacOS, with
        the ability to run virtually all of the programs expected from a desktop
        computer.
    Linux operating systems are also a popular choice for web servers and have
        become the backbone of enterprise computing \cite{grandview}.
    According to the Linux Foundation, Linux powers the majority of the public
        cloud \cite{linux-public-cloud}, and some companies, like Amazon Web
        Services, have developed their own Linux distribution for use in their
        products \cite{aws-linux}.

    Most pertinent to this report, however, is the use of Linux in embedded
        applications.
    While some embedded devices do not employ an operating system (these are
        called \textit{bare-metal} systems, and they forgo an operating system
        to conserve resources \cite{bare-metal}), most are complex enough to
        require an operating system.
    And when an embedded system requires an operating system, Linux is an
        appropriate choice for its kernel.

    The Linux Foundation estimates that 62\% of embedded systems use a
        Linux-based operating system \cite{linux-public-cloud}.
    This is possible because of the high degree of modularity within the Linux
        kernel, making it easy to configure the kernel to specific hardware
        \cite{embedded-textbook}.
    Further, developers of embedded Linux distributions have the ability to
        exclude packages specific to desktops, like user systems and GUI
        environments, opting instead for packages suited for embedded development,
        like cross-development tools, different types of drivers, and debugging
        and profiling tools \cite{embedded-textbook}.
    This extensibility and flexibility makes embedded Linux a great choice.
